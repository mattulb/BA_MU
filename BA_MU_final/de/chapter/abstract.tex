Diese Bachelorarbeit untersucht den vielfältigen Einsatz homomorpher Kryptosysteme in Forschungsarbeiten. Dabei wird erörtert welches Kryptosystem zum Einsatz kommt und aus welchen Gründen sich für dieses entschieden wurde. Es wird untersucht in welchem Anwendungskontext das Kryptosystem zum Einsatz kommt und ob eine unautorisierte Verformbarkeit der Chiffretexte (engl. \textit{malleability}) bei der Implementierung berücksichtigt wurde. Unter der Verformbarkeit von Chiffretexten versteht man die Möglichkeit eines Angreifers mit verschlüsselten Zahlen eine Funktion wie die Addition zweier Zahlen auszuführen. Normalerweise würde ein Angreifer dazu beide Zahlen zunächst separat entschlüsseln müssen, wofür er einen geheimen Schlüssel als Eingabe für den Entschlüsselungsalgorithmus benötigt. Dann führt der Angreifer die Addition auf den Klartexten durch und verschlüsselt ihr Ergebnis wieder. Homomorphe Kryptosysteme ermöglichen Berechnungen mit verschlüsselten Zahlen wie z.B. deren Addition ohne Kenntnis eines privaten Schlüssels und ohne vorherige Entschlüsselung. Damit ist ein Angreifer bei einem homomorphen Kryptosystem in der Lage verschlüsselte Zahlen zu verändern. Findet diese Veränderung unbemerkt und unerlaubt statt, wird die Integrität von Rechenoperationen im Chiffreraum verletzt.

Die evaluierten Einsatzszenarien homomorpher Kryptosysteme werden im Einzelnen kurz vorgestellt um anschließend Kriterien zu erstellen anhand welcher eine Kategorisierung der Kryptosysteme erfolgt. Dies geschieht im Hinblick darauf Dritte bei der Entscheidung für den Einsatz homomorpher Kryptographie in eigenen Arbeiten zu unterstützen.
 