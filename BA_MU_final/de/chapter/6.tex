\chapter{Verwandte Arbeiten}

Im Folgenden wird werden mehrere verwandte Arbeiten kurz vorgestellt.

In \cite{maimut2012homomorphic} werden verschiedene semi- und vollhomorphe Kryptosysteme als generelle Lösung für Vertraulichkeitsprobleme im Cloud Computing genannt. Sie erwarten, dass homomorphe Kryptographie in der Zukunft effizienter sein wird und somit ihre Relevanz für den Praxiseinsatz steigt. In Abschnitt 3 werden erste Beispiele für einen Einsatz in Cloud Computing, sicheren Wahlsystemen und privaten Informationsrückgewinnungssystemen erwähnt.
Der Einsatz von homomorpher Kryptographie geschieht nur unter dem Hintergrund der Sicherung von Vertraulichkeit. Das Ausnutzen von einer Verformbarkeit der Chiffretexte durch einen Angreifer ist nicht Untersuchungsgegenstand.

Der Artikel \cite{fontaine2007survey} dient als Übersicht und Einführung in homomorphe Kryptographie für Wissenschaftler, die mit Kryptographie nicht vertraut sind. Die Autoren betonen ausdrücklich die Verformbarkeit von Chiffretexten in homomorphen Kryptosystemen und stellen klar, dass diese geforderte Eigenschaft das Erreichen einer Sicherheit gegen stärkere Angriffsmodelle in der Kryptoanalyse unmöglich macht. Sie gehen  darauf ein, dass die Forderung nach semantischer Sicherheit (\ref{semantischSicher}) zur Folge hat, dass eingesetzte Kryptosysteme probabilistisch sein müssen.
Das Ausnutzen von einer Verformbarkeit der Chiffretexte durch einen Angreifer wird auch hier nicht untersucht.

Beide Quellen nennen vollhomomorphe Kryptographie unpraktikabel für den praktischen Einsatz wegen hoher Anforderung an verfügbare Ressourcen.

Als relativ junger Teilbereich der Kryptographie lassen sich in den Veröffentlichungen verschiedene Definitionen zu vollhomomorphen Kryptosystemen finden. In \cite{armknecht2015guide} werden diese unter einem rein theoretischen Hintergrund systematisch gegliedert und Bezüge der verschiedenen Definitionen zueinander hergestellt. 