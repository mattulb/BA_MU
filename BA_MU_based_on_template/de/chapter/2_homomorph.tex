\chapter{Homomorphe Kryptosysteme}

\section{Schutzziele der Kryptografie}
Die Kryptografie möchte mehrere Schutzziele für die Speicherung, Vervielfältigung und Übertragung von Informationen umsetzen und dazu verschiedene Verfahren bereitstellen. Grundlegende Schutzziele beim Übertragen von Informationen zwischen mehreren Personen mit Nachrichten sind dann \cite[p.2]{delfs2002introduction}:
%\cite{mm2015itsec}
\begin{enumerate}
	\item \textbf{Vertraulichkeit:} Keine unauthorisierte Kenntnisnahme. Nur dazu berechtigte Personen sollen eine Information lesen können oder Zugang zur einer Information bekommen.
	\item \textbf{Integrität:} Keine unauthorisierte unbemerkte  Datenmanipulation. Dies schützt insbesondere vor dem Hinterlegen von Falschdaten in einer Nachricht oder einer oder dem Fehlen von Teilen einer Nachricht.
	\item \textbf{Authentizität:} Der Empfänger kann den Verfasser einer Nachricht verifizieren. 
	\item \textbf{Nichtabstreitbarkeit:} Der Sender kann dem Versand nicht mehr abstreiten der Verfasser einer Nachricht gewesen zu sein.
\end{enumerate}

\section{Kryptosysteme}
Ein Kryptosystem ist ein Sammlung von Algorithmen um das Schutzziel der Vertraulichkeit beim Übertragen von Informationen in Nachrichten umzusetzen.

Damit ermöglicht ein Kryptosystem zwei Parteien Alice und Bob über einen ungeschützten Kanal in dem die Nachricht übertragen wird zu kommunizieren, ohne das eine dritte Partei welche mithört Zugang zu dem Inhalt der Nachricht bekommt.

Die zugrundelegenden Algorithmen und resultierende Eigenschaften über die Beziehung von Klartexten zu Chiffretexten führen zu ein verschiedenen Klassen von Kryptosystemen. Diese Kryptosysteme führen wir in diesem Abschnitt ein. In der Literatur ist es üblich bei \enquote{einfacheren} Kryptosystem die deterministisch oder symmetrisch sind, diese Bezeichnungen wegzulassen. Zur besseren Abgrenzung werden in diesem Abschnitt Kryptosysteme immer mit ihren Eigenschaften genannt (z.B. deterministisches symmetrisches Kryptosystem).

Formal definieren wir \cite[p.1]{stinson2006cryptography}:

\newtheorem{theorem}{Definition}[section]
\begin{theorem}[Kryptosytem]
	\label{KS}
	Ein Kryptosystem ist ein Quintupel $(\mathcal{P},\mathcal{C},\mathcal{K},\mathcal{E},\mathcal{D})$ welches folgenden Eigenschaften genügt:
	\begin{enumerate}
		\item $\mathcal{P}$ ist eine endliche Menge von Klartexten, der Klartextraum.
		\item $\mathcal{C}$ ist eine endliche Menge von Chiffretexten, der Chiffreraum.
		\item $\mathcal{K}$ ist eine endliche Menge möglicher Schlüssel, der Schlüsselraum.
		\item Für alle Schlüssel $k\in \mathcal{K}$ gibt es eine Verschlüsselungsfunktion $\mathcal{E}\ni e_k:\mathcal{P}\rightarrow\mathcal{C}$ und zugehörige Entschlüsselungsfunktion $\mathcal{D}\ni d_k:\mathcal{C}\rightarrow\mathcal{D}$, so dass für alle Klartexte $x\in\mathcal{P}$ folgende Identität gilt: $d_k(e_k(x)) = x$
	\end{enumerate}
\end{theorem}

Grundsätzlich gibt es also drei Algorithmen in einem Kryptosystem: Einen für die Erzeugung des Schlüssels, einen für die Verschlüsselung und einen für die Entschlüsselung \cite{Cryptosy29:online}.

\begin{theorem}[Symmetrisches Kryptosystem]
	Sei $K = (\mathcal{P},\mathcal{C},\mathcal{K},\mathcal{E},\mathcal{D})$ ein Kryptosystem. Dann nennen wir K symmetrisch, wenn sowohl die Verschlüsselungs-funktion $e_k$ als auch die Entschlüsselungsfunktion $d_k$ vollständig von \textit{demselben} Schlüssel $k\in\mathcal{K}$ abhängen. Vollständig bedeutet, dass diese Funktionen insbesondere nicht nur von einen Teilschlüssel von $k$ anhängen, wenn der Schlüssel k aus mehreren Parametern zusammengesetzt ist. Letzteres ist in \ref{KS} nicht vorrausgesetzt.
\end{theorem}

Ein Nachteil von Symmetrischen Kryptosystemen liegt auf der Hand: Da sowohl Alice als auch Bob den gleichen geheimen Schlüssel benötigen, muss dieser über einen sicheren Kanal übertragen werden. Daher sind  symmetrische Kryptosysteme auch bekannt als private-key Kryptosysteme.

Im Gegensatz dazu gibt es Kryptosysteme bei denen sich $K$ aus einem privaten und öffentlichen Teilschlüssel zusammensetzt von Alice den öffentlichen Teilschlüssel bekannt geben kann um dritten zu ermöglichen ihr Informationen vertraulich zukommen zu lassen. Öffentlicher und privater Teilschlüssel stehen in einem Zusammenhang, der jedoch für Angreifer mit begrenzter Rechenkapazität nicht möglich ist zu erschließen.

\begin{theorem}[Asymmetrisches Kryptosystem]
	Sei $K = (\mathcal{P},\mathcal{C},\mathcal{K},\mathcal{E},\mathcal{D})$ ein Kryptosystem. Dann nennen wir K asymmetrisch wenn sich der Schlüssel $k\in\mathcal{K}$ zusammensetzt aus $k=(k_s, k_p)$. Die Verschlüsselungsfunktion ist dann $\mathcal{E}\ni e_{k_p}:\mathcal{P}\rightarrow\mathcal{C}$, während die Entschlüsselungsfunktion $\mathcal{D}\ni d_{k_s}:\mathcal{P}\rightarrow\mathcal{C}$ ist. Während $e_k$ von beliebigen Parteien ausgeführt werden kann, kann $d_k$ nur vom Besitzer des privaten Teilschlüssel $k_s$ ausgeführt werden. $k_s$ muss geheim gehalten werden.
\end{theorem}

Diese drei Definitionen genügen noch nicht um zu beschreiben, in welcher Beziehung Klartexte $x$ zu ihren Chiffraten $c$ stehen wenn sie mit dem gleichen Schlüssel $k$ in verschiedenen Ausführungen von $e_k$ erzeugt werden. Dies ist von Bedeutung für mögliche Angriffe der Kryptoanalyse welche in \ref{Sicherheitskriterien} vorgestellt werden.

\begin{theorem}[Deterministisches Kryptosytem]\footnote{Diese Definition ist eine Formalisierung von saloppen Beschreibungen in der Literatur.}
	Sei $K = (\mathcal{P},\mathcal{C},\mathcal{K},\mathcal{E},\mathcal{D})$ ein Kryptosystem. Dann nennen wir K deterministisch wenn gilt: Für einen beliebigen festen Schlüssel $k\in\mathcal{K}$ ist $e_k$ ist injektiv. %Falls $e_k$ nicht injektiv ist, nennen wir K proballistisch.
\end{theorem}

Seien nun $c_1,c_2\in\mathcal{C}, c_1= c_2$ zwei Chiffrate unter $e_k$, dann folgt daraus für ihre Klartexte, dass $x_1= x_2$. Also führt der gleiche Klartext unter Verwendung desselben Schlüssel bei verschiedenen Ausführungen von der Veschlüsselungsfunktion $d_k$ zu einem identischen Chiffrat!

Jetzt können wir in Abgrenzung zu dieser Definition das Proballistische Kryptosystem einführen. Ein Proballistisches Kryptosystem erzeugt für gleiche Klartexte bei demselben Schlüssel mit jeder Ausführung der Verschlüsselungsfunktion ein
% im Allgemeinen
anderes Chiffrat.
% Wohlgemerkt: im Allgemeinen, d.h. nicht immer. Deswegen ist dieses Kryptosystem strenggenommen nicht gegensätzlich zu einem Deterministischen Kryptosystem definiert.

Das Konzept eines Proballistischen Kryptosystems wurde ursprünglich von Goldwasser und Micali eingeführt in \cite{goldwasser1984probabilistic}. Wir definieren in Anlehnung an \cite[p.345]{delfs2002introduction}:

\begin{theorem}[Proballistisches Kryptosytem]
	\label{PKS}
	Ein Proballistisches Kryptosystem ist ein Sextupel $(\mathcal{P},\mathcal{C},\mathcal{K},\mathcal{E},\mathcal{D},\mathcal{R})$. Wie schon in \ref{KS} ist $\mathcal{P}$ ist der Klartextraum, $\mathcal{C}$ der Chiffreraum und $\mathcal{K}$ der Schlüsselraum. Neu sind:
	\begin{itemize}
		\item $\mathcal{R}$ ist eine endliche Menge von Zufallszahlen
		\item Für alle Schlüssel $k\in \mathcal{K}$ gibt es eine Verschlüsselungsfunktion $\mathcal{E}\ni e_k:\mathcal{P}\times\mathcal{R}\rightarrow\mathcal{C}$ und zugehörige Entschlüsselungsfunktion $\mathcal{D}\ni d_k:\mathcal{C}\times\mathcal{R}\rightarrow\mathcal{D}$, so dass für alle Klartexte $x\in\mathcal{P}$ und alle Zufallszahlen $r\in\mathcal{R}$ folgende Identität gilt: $d_k(e_k(x,r)) = x$ 
		\item Für eine Zufallszahl $r\in\mathcal{R}$ und verschiedene Klartexte $x_1,x_2\in\mathcal{P}, x_1\neq x_2$ gilt: $e_k(x_1,r)\neq e_k(x_2,r)$.
		
	\end{itemize}
\end{theorem}

Ein Proballistisches Kryptosystem benutzt also Zufall in der Verschlüsselungs-funktion, so dass der gleiche Klartext verschieden verschlüsselt wird. Mit Proballistischen Kryptosystemen werden meistens Asymmetrische Verschlüsselungs-verfahren gemeint, es ist jedoch auch möglich mit Symmetrischen Verschlüsselungs-verfahren diese Eigenschaft zu erreichen, z.B. bei Verwendung von Blockchiffren im Cipher Block Chaining Mode. Die Menge der Zufallszahlen $\mathcal{R}$ entspricht dann der Menge möglicher Initialisierungsvektoren. 
%\cite{mm2015itsec}. 




\subsubsection{Homomorphes Kryptosystem}

\subsubsection{Stufenfixes Homomorphes Kryptosystem}
\label{LHE}

\subsection{Mathematische Grundlagen}
\label{MG}

Wir werden später Homomorphe Kryptosystem im Detail einführen. Um ihre Anwendung zu verstehen, ist es jedoch nötig folgende Algebraische Strukturen nach \cite{tr2015la} einzuführen um ein Verständnis dafür zu schaffen wie Homomorphe Kryptosysteme verwendet werden könnnen.

\begin{theorem}[Gruppe]
	Eine Gruppe ist ein Tupel $(G,+)$ bestehend aus der Menge $G$ und einer Verknüpfung $+$ auf $G$ mit folgenden Eigenschaften:
	\begin{itemize}
		\item $+$ ist assoziativ
		\item Es existiert bzgl. $+$ ein neutrales Element $e$ in $G$.
		\item Jedes $g$ in $G$ ist invertierbar.
	\end{itemize}
	
	Ist die Verknüpfung einer Gruppe zusätzlich kommutativ, so nennt man sie abelsch.
	
\end{theorem}

\begin{theorem}[Ring]
	\label{Ring}
	Ein Ring ist ein Tripel $(R,+,\cdot)$ bestehend aus der Menge $R$ und zwei Verknüpfungen $+$ und $\cdot$ auf $R$ mit folgenden Eigenschaften:
	\begin{itemize}
		\item R ist bzgl. $+$ eine abelsche Gruppe.
		\item $\cdot$ ist assoziativ
		\item Es gelten die Distributivgesetze: $\forall a,b,c\in R: a\cdot(b+c)=(a\cdot b)+(a\cdot c)$ und $(a+b)\cdot c = (a\cdot c)+(b\cdot c)$
	\end{itemize}
\end{theorem}

Hat $R$ bzgl. $\cdot$ ein neutrales Element, so nennen wir $R$ einen \enquote{Ring mit Rins}. Ist $R$ bzgl. $\cdot$ kommutativ, so nennen wir $R$ einen \enquote{kommutativen Ring}. 

\begin{theorem}[Körper]
	Sei $K$ ein kommutativer Ring mit Eins wie in \ref{Ring}, so heißt $K$ Körper, wenn die neutralen Elemente bzgl. der Verknüpfungen $+$ und $\cdot$ verschieden sind und alle Elemente bzgl. $\cdot$ invertierbar sind.
	
\end{theorem}

Um den Homomorphismus Einführen zu können benötigen wir eine noch mächtigere algebraische Struktur.

\begin{theorem}[K-Vektorraum]
	Sei $(K,+,\cdot)$ ein Körper und $V$ eine Menge. Zusätzlich existieren zwei Verknüpfungen $\oplus: V\times V \rightarrow V$ und $\oplus: K\times V \rightarrow V$, so dass gilt:\footnote{Evtl. ausführlicher formulieren}
	\begin{itemize}
		\item $\oplus$ ist assoziativ, kommutativ, hat ein neutrales Element bzgl. $V$ und für alle Elemente $v\in V$ Inverse in $V$
		\item $\otimes$ ist assoziativ, hat ein neutrales Element bzgl. $V$
		\item Elemente aus $K$ und $V$ sind distributiv
	\end{itemize}
	Dann nennen wir $V$ einen K-Vektorraum.
\end{theorem}

\begin{theorem}[Homomorphismus]
	Seien $V,W$ zwei K-Vektorräume und $f:V\rightarrow W$ eine Abbildung. Dann nennen wir $f$ einen Homomorphismus (oder linear) von $V$ nach $W$ wenn gilt:
	\begin{itemize}
		\item $\forall x,y\in V : f(x)+f(y) = f(x+y)$
		\item $\forall x\in V, \lambda\in K: f(\lambda\cdot K)=\lambda\cdot f(x)$
	\end{itemize}
\end{theorem}

\newtheorem{theorem2}{Theorem}[section]
\begin{theorem2}[Jede boolsche Funktion ist mit NANDs konstruierbar]
	$\ldots$
\end{theorem2}

Homomorphes Kryptosystem:
Eine Verknüfpung im Chiffreraum führt zur einer Verknüpfung im Klartextraum. Erstere ist in der Regel die Multiplikation und letztere Addition, Multiplikation oder XOR.