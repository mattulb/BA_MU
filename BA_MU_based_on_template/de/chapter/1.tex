\chapter{Einleitung}

Ziel dieser Arbeit ist eine Erörterung der Vorteile von Anwendungen homomorpher Kryptosysteme für praktische Anwendungen. Homomorphe Kryptosysteme gliedern sich ihrerseits in semihomomorphe und vollhomomorphe Kryptosysteme. Während semihomomorphe Kryptosysteme eingeschränkt Operationen auf Chriffren ermöglichen - abhängig von dem gewählten Kryptosystem, ermöglichen vollhomomorphe Kryptosysteme die Berechnung jeder booleschen Funktion im Chiffreraum. Da letztere jedoch aufgrund hoher Laufzeiten weniger praktikabel sind greift man in der Praxis fast ausschließlich auf semihomomorphe Kryptosysteme zurück. Dies führt in der Praxis zu verschiedenen, vom jeweiligen Anwendungsfall abhängigen Einsatz eines Kryptosystems.

\section{Beitrag dieser Bachelorarbeit}

Es ist von vornherein nicht klar welche Gründe jemand gewählt hat ein Kryptosystem dem anderen vorzuziehen, insbesondere wenn sie den gleichen Operator im Chiffreraum zur Verfügung stellen. Weiter lässt sich mit einem Kryptosystem wie Paillier der Operator von Goldwasser-Micali simulieren. Diese Arbeit will die Gründe für den Einsatz verschiedener homomorphen Kryptosysteme untersuchen und Informationen sammeln um die einzelnen Kryptosysteme zu kategorisieren als Referenz für den zukünfigen Einsatz in Forschungsarbeiten.

Es wird untersucht wie mit der Malleabliltät der homomorphen Kryptosysteme im einzelnen Umgegangen wird um die Möglichkeit einer Manipulation der Daten auszunutzen die einen Angreifer ermöglichen mehr Information aus dem System abzuschöpfen. 

\section{Aufbau}

Zunächst werden im Kapitel \ref{HK} der Begriff eines Kryptosystems und darauf aufbauende Erweiterungen bis hin zum probabilistischen asymmetrischen Kryptosystem erläutert welches die Grundlage vieler semihomomorpher Verfahren bildet. Dann werden in Kapitel \ref{SK} verschiedene Sicherheitskriterien erläutert die in den untersuchten Anwendungsfällen entweder realisiert werden oder zum Verständnis benötigt werden. Einige dieser Sicherheiskriterien sind insbesondere von Notwendigkeit für die Klassifierung. In Kapitel \ref{KHK} werden die untersuchten Studien vorgestellt, allerdings nicht in ihrem vollem Umfang, sondern in Bezug auf den Einsatz homomorpher Kryptosysteme, der Gründe für den Einsatz und dem Umgang mit Malleablilität.