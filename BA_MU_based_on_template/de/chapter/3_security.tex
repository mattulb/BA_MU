\chapter{Sicherheitskriterien}
\label{Sicherheitskriterien}
\section{Malleability}
Malleability beschreibt die Möglichkeit, dass ein Angreifer einen Chiffretext $c$ von Klartext $k$ geziehlt verformen kann um einen daraus abgeleiteten Chiffretext $f(c)=c'$ zu erzeugen welcher in einer ihm bekannten Beziehung $f$ zu $c$ steht. Existiert nun zwischen den Klartexten $k$ und $k'$ eine Beziehung die der Angreifer umkehren kann, kann er zu $k'$ den ursprünglichen Klartext $k$ bestimmten. \cite[p. 292]{smart2003}

\textbf{Eigenschaften}: Ein malleable Kryptosystem ist angreifbar mit chosen ciphertext Angriffen. (CCA2)

\textbf{Kommentar}: Das zwischen den Chiffretexten und den Klartexten eine ähnliche Beziehung steht die der Angreifer kennt ist eine Eigenschaft die genau Isomorphismen ermöglichen! Daher sind homomorphe Kryptosysteme per Design anfällig für malleability.

\section{Privacy-Preserving}

\section{Sicherheitsklassen}
\subsection {Ununterscheidbarkeit von Geheimtexten (Ciphertext Indistinguishability)}
\subsection{Semantische Sicherheit}
Ein deterministisches Kryptosytem wie in \ref{DKS} kann nie semantisch sicher sein!