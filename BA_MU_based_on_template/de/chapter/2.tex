\chapter{Homomorphe Kryptosysteme}
\label{HK}

\section{Schutzziele der Kryptografie}
Die Kryptografie hat zur Aufgabe Lösungen für die Realisierung verschiedener Schutzziele bei der Speicherung, Vervielfältigung und Übertragung von Informationen umsetzen. Die Kryptografie stellt dazu verschiedene Algorithmen und Protokolle bereit. Grundlegende Schutzziele beim Übertragen von Informationen zwischen mehreren Parteien in  Nachrichten sind nach \cite[p.4]{menezes1996handbook}\cite[p.2]{delfs2002introduction}:
%\cite{mm2015itsec}
\begin{enumerate}
	\item \textbf{Vertraulichkeit:} Keine unauthorisierte Kenntnisnahme. Nur dazu berechtigte Personen sollen eine bestimmte Information lesen können oder Zugang zu dieser Information erhalten. Dieser Begriff ist Synonym mit Geheimhaltung. Vertraulichkeit kann physisch erreicht werden oder durch mathematische Algorithmen welche die Daten unverständlich machen. 
	\item \textbf{Integrität:} Keine unauthorisierte unbemerkte Datenmanipulation. Um die Integrität von Daten zu gewährleisten müssen die Möglichkeit einer Detektion von Veränderungen in den Daten realisiert werden.  Dies schützt insbesondere vor dem Hinterlegen von Falschdaten in einer Nachricht oder dem Fehlen von Teilen einer Nachricht.
	\item \textbf{Authentizität:} Authentizität meint die Fähigkeit einer Identifikation in Bezug auf die Information als auch auf die Kommmunikationspartner (Entitäten). Fordert man, dass die kommunizierenden Teilnehmer in der Lage sein sollen sich gegenseitig zu identifizieren spricht man von Authentizität der Entität. Bei einseitiger Kommunikation fordert man lediglich, den Urheber einer Nachricht identifizieren zu können, also die Authentizität des Datenursprungs.
	\item \textbf{Nichtabstreitbarkeit:} Der Versand einer Nachricht kann von dem Sender nach dem Versand nicht mehr abgestritten werden. Zum Beispiel wenn eine Entität den Kauf in einer unabstreitbaren Nachricht zunächst authorisiert, aber später verneint, so kann ihr Konfliktfällen die ursprüngliche Zusage nachgewiesen werden.
\end{enumerate}

\section{Kryptosysteme}
Ein Kryptosystem ist ein Sammlung von Algorithmen um das Schutzziel der Vertraulichkeit bei der Übertragung von Informationen umzusetzen.

Damit ermöglicht ein Kryptosystem zwei Parteien Alice und Bob über einen ungeschützten Kanal in dem die Nachricht übertragen wird zu kommunizieren, ohne das eine dritte Partei Zugang zu der geschützten der Information erhält.

Die zugrundelegenden Algorithmen und resultierende Eigenschaften über die Beziehung von Klartexten zu Chiffretexten führen zu verschiedenen Klassen von Kryptosystemen. Diese Kryptosysteme führen wir in diesem Abschnitt ein. In der Literatur ist es üblich bei \enquote{einfacheren} Kryptosystem die deterministisch oder symmetrisch sind, diese Bezeichnungen wegzulassen. Zur besseren Abgrenzung werden in diesem Abschnitt Kryptosysteme immer mit ihren Eigenschaften genannt (z.B. deterministisches symmetrisches Kryptosystem).

In der Literatur lassen sich verschiedene Formalisierungen für ein Kryptosystem finden, und es ist nicht trivial sie zu harmonisieren. Wir benutzten die Definition von Douglas R. Stinson \cite[p.1]{stinson2006cryptography} und erweitern diese Definition im Anschluss für eine bessere Differenzierung um Eigenschaften wie Determinismus oder Symmetrie.

\newtheorem{theorem}{Definition}[section]
\begin{theorem}[Kryptosytem]
	\label{KS}
	Ein Kryptosystem ist ein Quintupel $(\mathcal{P},\mathcal{C},\mathcal{K},\mathcal{E},\mathcal{D})$ welches folgenden Eigenschaften genügt:
	\begin{enumerate}
		\item $\mathcal{P}$ ist eine endliche Menge von Klartexten, der Klartextraum.
		\item $\mathcal{C}$ ist eine endliche Menge von Chiffretexten, der Chiffreraum.
		\item $\mathcal{K}$ ist eine endliche Menge möglicher Schlüssel, der Schlüsselraum.
		\item Für alle Schlüssel $k\in \mathcal{K}$ gibt es eine Verschlüsselungsfunktion $\mathcal{E}\ni e_k:\mathcal{P}\rightarrow\mathcal{C}$ und zugehörige Entschlüsselungsfunktion $\mathcal{D}\ni d_k:\mathcal{C}\rightarrow\mathcal{D}$, so dass für alle Klartexte $x\in\mathcal{P}$ folgende Identität gilt: $d_k(e_k(x)) = x$
	\end{enumerate}
\end{theorem}

Grundsätzlich muss ein Kryptosystem also mindestens drei Algorithmen bereitstellen: Einen für die Erzeugung des Schlüssels, einen für die Verschlüsselung und einen für die Entschlüsselung \cite{Cryptosy29:online}. Man beachte, dass der Schlüssel $k$ als \textit{ein} Element des Schlüsselraums selber aus \textit{mehreren} Elementen zusammengesetzt werden kann. Diese Eigenschaft führt zu den nächsten beiden Erweiterungen.

\begin{theorem}[Symmetrisches Kryptosystem]
	Sei $K = (\mathcal{P},\mathcal{C},\mathcal{K},\mathcal{E},\mathcal{D})$ ein Kryptosystem. Dann nennen wir K symmetrisch, wenn jede Verschlüsselungsfunktion $e_k$ als auch die zugehörige Entschlüsselungsfunktion $d_k$ vollständig von \textit{demselben} Schlüssel $k\in\mathcal{K}$ abhängen. Vollständig bedeutet, dass diese Funktionen insbesondere nicht nur von einer Teilmenge von $k$ abhängen, wenn der Schlüssel k aus mehreren Parametern zusammengesetzt ist. Alle Parameter von $k$ gehen in die Verschlüsslungsfunktion und Entschlüsselungsfunktion ein. Letzteres ist in \ref{KS} nicht vorrausgesetzt.
\end{theorem}

Ein Nachteil von Symmetrischen Kryptosystemen liegt auf der Hand: Da sowohl Alice als auch Bob den gleichen geheimen Schlüssel benötigen, muss dieser über einen sicheren Kanal übertragen werden bevor sie geheime Nachrichten austauschen können. Daher sind  symmetrische Kryptosysteme auch bekannt als private-key Kryptosysteme.

Im Gegensatz dazu gibt es Kryptosysteme bei denen $k$ aus einem privaten und einem öffentlichen Teilschlüssel zusammensetzt ist. Diese Teilmengen müssen nicht disjunkt\footnote{In RSA enthalten die Mengen beider Teilschlüssel den Modulus $n$ \cite[p.6]{rivest1978method}.} sein. Alice kann nun ihren öffentlichen Teilschlüssel bekannt geben um so dritten zu ermöglichen ihr Informationen vertraulich zukommen zu lassen. Daher spricht man auch von public-key Kryptosystemen, ein Konzept das Ursprünglich von Diffie und Hellmann in \cite[p.648]{diffie1976new} eingeführt wurde.

\begin{theorem}[Asymmetrisches Kryptosystem]
	Sei $K = (\mathcal{P},\mathcal{C},\mathcal{K},\mathcal{E},\mathcal{D})$ ein Kryptosystem. Dann nennen wir K asymmetrisch wenn sich der Schlüssel $k\in\mathcal{K}$ zusammensetzt aus $k=(k_s, k_p)$ mit $k_s,k_p\in K$. Die Verschlüsselungsfunktion ist dann $\mathcal{E}\ni e_{k_p}:\mathcal{P}\rightarrow\mathcal{C}$, während die Entschlüsselungsfunktion $\mathcal{D}\ni d_{k_s}:\mathcal{P}\rightarrow\mathcal{C}$ ist. Während $e_k$ von beliebigen Parteien ausgeführt werden kann, kann $d_k$ nur vom Besitzer des privaten Teilschlüssel $k_s$ ausgeführt werden. $k_s$ muss geheim gehalten werden.
\end{theorem}

Öffentlicher und privater Teilschlüssel stehen in einem mathematischen Zusammenhang, der jedoch für Angreifer mit begrenzter Rechenkapazität praktisch unmöglich ist zu erschließen.

Diese drei Definitionen genügen noch nicht um zu beschreiben, in welcher Beziehung Klartexte $x$ zu ihren Chiffraten $c$ stehen wenn sie mit dem gleichen Schlüssel $k$ in verschiedenen Ausführungen von $e_k$ erzeugt werden. Dies ist von Bedeutung für mögliche Sicherheitsklassen welche in \ref{Sicherheitsklassen} vorgestellt werden.

\begin{theorem}[Deterministisches Kryptosytem]
%\footnote{Diese Definition ist eine Formalisierung von saloppen Beschreibungen in der Literatur.}
	Sei $K = (\mathcal{P},\mathcal{C},\mathcal{K},\mathcal{E},\mathcal{D})$ ein Kryptosystem. Dann nennen wir K deterministisch wenn gilt: Für einen beliebigen festen Schlüssel $k\in\mathcal{K}$ ist $e_k$ injektiv. %Falls $e_k$ nicht injektiv ist, nennen wir K proballistisch.
\end{theorem}

Seien nun $c_1,c_2\in\mathcal{C}, c_1= c_2$ zwei Chiffrate unter $e_k$, dann folgt daraus für ihre Klartexte, dass $x_1= x_2$. Also führt der gleiche Klartext unter Verwendung desselben Schlüssel bei verschiedenen Ausführungen von der Verschlüsselungsfunktion $d_k$ zu einem identischen Chiffrat!

Jetzt können wir in Abgrenzung zu dieser Definition das probabilistische Kryptosystem einführen. Ein Proballistisches Kryptosystem erzeugt für gleiche Klartexte bei demselben Schlüssel mit jeder Ausführung der Verschlüsselungsfunktion ein
% im Allgemeinen
anderes Chiffrat.
% Wohlgemerkt: im Allgemeinen, d.h. nicht immer. Deswegen ist dieses Kryptosystem strenggenommen nicht gegensätzlich zu einem Deterministischen Kryptosystem definiert.

Das Konzept eines Probalistischen Kryptosystems wurde ursprünglich von Goldwasser und Micali eingeführt in \cite{goldwasser1984probabilistic}. Wir definieren in Anlehnung an \cite[p.345]{stinson2006cryptography}:

\begin{theorem}[Probalistisches Kryptosytem]
	\label{PKS}
	Ein Probalistisches Kryptosystem ist ein Sextupel $(\mathcal{P},\mathcal{C},\mathcal{K},\mathcal{E},\mathcal{D},\mathcal{R})$. Wie bereits in Definition \ref{KS} ist $\mathcal{P}$ ist der Klartextraum, $\mathcal{C}$ der Chiffreraum und $\mathcal{K}$ der Schlüsselraum. Neu sind:
	\begin{itemize}
		\item $\mathcal{R}$ ist eine endliche Menge von Randomisierern
		\item Für alle Schlüssel $k\in \mathcal{K}$ gibt es eine Verschlüsselungsfunktion $\mathcal{E}\ni e_k:\mathcal{P}\times\mathcal{R}\rightarrow\mathcal{C}$ und zugehörige Entschlüsselungsfunktion $\mathcal{D}\ni d_k:\mathcal{C}\times\mathcal{R}\rightarrow\mathcal{D}$, so dass für alle Klartexte $x\in\mathcal{P}$ und alle Randomisierer $r\in\mathcal{R}$ folgende Identität gilt: $d_k(e_k(x,r)) = x$ 
	\end{itemize}
	Für ein festes $k\in K$ und ein beliebigen Klartext $x\in P$ definieren wir die Wahrscheinlichkeitsverteilung $p_{K,x}$ auf $C$, so dass  $p_{K,x}(y)$ die Wahrscheinlichkeit angibt, dass $y$ ein Chiffrat von $x$ unter $e_k$ ist. Nun fordern wir:
	
	\begin{itemize}
		\item Gegeben $x_1,x_2\in P, x_1\neq x_2$ und $k\in K$. Dann sind die Wahrscheinlichkeitsverteilungen $p_{K,x_1}$ und  $p_{K,x_2}$ nicht in Polynomialzeit unterscheidbar.
	\end{itemize}
\end{theorem}

Nicht unterscheidbar hat insbesondere zur Folge, dass wir auch nicht wissen ob die gleiche Wahrscheinlichkeitsverteilung hinter zwei Chiffraten steckt. In anderen Worten: Die wiederholte Verschlüsselung eines Klartextes führt im Allgemeinen zu verschiedenen Chiffraten.

Ein Proballistisches Kryptosystem nutzt Zufall in der Verschlüsselungsfunktion, so dass der gleiche Klartext verschieden verschlüsselt wird. Mit Proballistischen Kryptosystemen werden meistens Asymmetrische Verschlüsselungsverfahren gemeint, es ist jedoch auch möglich mit Symmetrischen Verschlüsselungsverfahren diese Eigenschaft zu erreichen, z.B. bei Verwendung von Blockchiffren im Cipher Block Chaining Mode. Die Menge der Randomizierer $\mathcal{R}$ entspricht dann der Menge möglicher Initialisierungsvektoren. 

\subsubsection{Semihomomorphes Kryptosystem}

Zusammen mit \ref{MG} können wir nun ein semihomomorphe Kryptosystem definieren. Laut \ref{Homomorphismus} ist ein Homomorphismus Strukturerhaltend. Für Verknüpfungen im Chiffreraum findet man eine entsprechende Verknüpfung im Klartextraum. Diese Eigenschaft ist \textit{die} Stärke der homomorphen Kryptographie, denn sie ermöglicht eine Verarbeitung von Daten unter Wahrung der Vertraulichkeit dieser und wird somit oft einsetzt um bekannte Protokolle privacy-preserving zu machen wie mehrere der vorgestellten Projekte in \ref{KHK} zeigen. Auf der anderen Seite kann die Homomorphieeigenschaft bösartig ausgenutzt werden wie in \ref{malleability} an einem Beispiel erläutert wird.

\begin{theorem}[Semihomomorphes Kryptosytem]
	\label{PKS}
	Sei $K = (\mathcal{P},\mathcal{C},\mathcal{K},\mathcal{E},\mathcal{D})$ ein asymmetrisches Kryptosystem. Wir nennen K semihomomorph, wenn $(P,\otimes)$ und $(C,\odot)$ Gruppen bilden und die Verschlüsselungsfunktion ein Homomorphismus von Gruppen ist. 
	\begin{itemize}
		\item Das heißt alle unter $e_{k_p}$ erzeugten Chiffrate bilden einen Gruppe.
		\item Sei $e_{k_p}(x_1)=c_1,e_{k_p}(x_2)=c_2$. Dann gilt: $d_{k_s}(c_1\odot c_2)= m_1\otimes m_2$
	\end{itemize}
\end{theorem}

Diese Definition ist orientiert an \cite[p.499]{katz2014introduction}.

\subsubsection{Vollhomomorphes Kryptosystem}

Lange Zeit nahm man an, dass es kein Kryptosysteme gibt, welches beliebige Rechenoperationen im Chiffreraum ermöglicht - d.h. vollhomomorph wäre. Bekannte semihomomorphe Kryptosysteme erlauben nur eingeschränkte Operationen(z.B. Addition oder XOR). Im Jahre 2009 stellte schließlich Craig Gentry \cite{gentry2009fully} erstmals ein Verfahren vor, wo im Chiffreraum alle Operationen eines Rings möglich sind, d.h. Addition und Multiplikation. Analog definieren wir nun \cite[p.47]{yi2014homomorphic}:

\begin{theorem}[Vollhomomorphes Kryptosytem]
	\label{PKS}
	Sei $K = (\mathcal{P},\mathcal{C},\mathcal{K},\mathcal{E},\mathcal{D})$ ein asymmetrisches Kryptosystem. Wir nennen K vollhomomorph, wenn $(P,\otimes,\ominus)$ und $(C,\odot,\oplus)$ Ringe bilden und die Verschlüsselungsfunktion ein Homomorphismus von Ringen ist. 
	\begin{itemize}
		\item Das heißt alle unter $e_{k_p}$ erzeugten Chiffrate bilden einen Ring.
		\item Sei $e_{k_p}(x_1)=c_1,e_{k_p}(x_2)=c_2$. Dann gilt: $d_{k_s}(c_1\odot c_2)= m_1\otimes m_2$ sowie $d_{k_s}(c_1\oplus c_2)= m_1\ominus m_2$
	\end{itemize}
\end{theorem}


Gentry konnte zeigen, dass ein beliebe Schaltung aus NANDs im Chiffreraum evaluiert werden kann. Dies ist ein Durchbruch, denn NANDs bilden für sich ein vollständiges Operatorensystem mit dem jede boolesche Funktion beschrieben werden kann \cite[p.129]{hoffmann2010grundlagen}. Damit ermöglicht ein vollhomomorphes Kryptosystem die Berechnung einer \textit{beliebigen} booleschen Funktion im Chiffreraum!

Ein vollhomomorphes Kryptosystem hat besteht aus vier Algorithmen. Zusätzlich zum Schlüsselgenerierung, Verschlüsselung und Entschlüsselung gibt es jetzt einen Algorithmus für die Evaluierung eines Schaltkreises im Chiffreraum. Sei $k_p$ der öffentliche Schlüssel unter dem die Chiffrate $c_1,\ldots,c_l$ erzeugt worden. Dann ermöglicht der Evaluierungsalgorithmus das Berechnen eines beliebigen Schaltkreises $S$ zu dem Ergebnis $c$:

\begin{equation*}
\textbf{Evaluate}_{k_p}(S,c_1,\ldots,c_l) = c
\end{equation*}

Trotz dieser großen Errungenschaft, finden bis heute vollhomomorphe Kryptosysteme wenig Einsatz in der Praxis, da die Verknüpfungen im Chiffreraum zu viel Rechenkapazität kosten. 


Statt drei jetzt vier algorithmen: KeyGen, Encrypt, Decrpyt, und Evaluate (homorphic evaluation) \cite{homomrphic encryption shcemes and applications for a secure digital worlds}

expansion rate
Ratio between lenght of ciphertext and the length of cleartext. \cite[p.63]{naccache1998new}

The expansion rate of probalistic cryptosystems is usually very large, i.e. one kilobit is needed to encrypt on a few bits.  \cite[p.60]{naccache1998new}

Stattdessen weicht man auf Semihomomorphe Kryptosysteme auf, wenn sie den Anforderungen genügen oder realisiert eine Kombination aus zwei Kryptosystemen von denen eins additiv und das andere multiplikativ ist. Ein Beispiel für so eine Realisierung werden wir in \ref{autocrypt} sehen.


\textbf{Abschließende Anmerkungen:}\\
Es kommen fast ausschließlich probalistische homomorphe Kryptosysteme zum Einsatz, obwohl es ebenso deterministische homomorphe Kryptosysteme gibt. Jedoch wurde bereits von Beoneh und Lipton in 1996 gezeigt, das \textit{jedes} deterministische homomorphe Kryptosystem in subexponentieller Zeit gebrochen werden kann \cite{boneh1996algorithms}. 

\subsubsection{Stufenfixes Homomorphes Kryptosystem}
\label{LHE}

\subsection{Algebraische Strukturen}
\label{MG}

Wir werden später Homomorphe Kryptosystem im Detail einführen. Um ihre Anwendung zu verstehen, ist es jedoch nötig folgende Algebraische Strukturen nach \cite{tr2015la} einzuführen um ein Verständnis dafür zu schaffen wie Homomorphe Kryptosysteme verwendet werden könnnen.

\begin{theorem}[Gruppe]
	Eine Gruppe ist ein Tupel $(G,+)$ bestehend aus der Menge $G$ und einer Verknüpfung $+$ auf $G$ mit folgenden Eigenschaften:
	\begin{itemize}
		\item $+$ ist assoziativ
		\item Es existiert bzgl. $+$ ein neutrales Element $e$ in $G$.
		\item Jedes $g$ in $G$ ist invertierbar.
	\end{itemize}
	
	Ist die Verknüpfung einer Gruppe zusätzlich kommutativ, so nennt man sie abelsch.
	
\end{theorem}

\begin{theorem}[Ring]
	\label{Ring}
	Ein Ring ist ein Tripel $(R,+,\cdot)$ bestehend aus der Menge $R$ und zwei Verknüpfungen $+$ und $\cdot$ auf $R$ mit folgenden Eigenschaften:
	\begin{itemize}
		\item R ist bzgl. $+$ eine abelsche Gruppe.
		\item $\cdot$ ist assoziativ
		\item Es gelten die Distributivgesetze: $\forall a,b,c\in R: a\cdot(b+c)=(a\cdot b)+(a\cdot c)$ und $(a+b)\cdot c = (a\cdot c)+(b\cdot c)$
	\end{itemize}
\end{theorem}

Hat $R$ bzgl. $\cdot$ ein neutrales Element, so nennen wir $R$ einen \enquote{Ring mit Rins}. Ist $R$ bzgl. $\cdot$ kommutativ, so nennen wir $R$ einen \enquote{kommutativen Ring}. 

\begin{theorem}[Körper]
	Sei $K$ ein kommutativer Ring mit Eins wie in \ref{Ring}, so heißt $K$ Körper, wenn die neutralen Elemente bzgl. der Verknüpfungen $+$ und $\cdot$ verschieden sind und alle Elemente bzgl. $\cdot$ invertierbar sind.
	
\end{theorem}

Um den Homomorphismus Einführen zu können benötigen wir eine noch mächtigere algebraische Struktur.

\begin{theorem}[K-Vektorraum]
	Sei $(K,+,\cdot)$ ein Körper und $V$ eine Menge. Zusätzlich existieren zwei Verknüpfungen $\oplus: V\times V \rightarrow V$ und $\oplus: K\times V \rightarrow V$, so dass gilt:\footnote{Evtl. ausführlicher formulieren}
	\begin{itemize}
		\item $\oplus$ ist assoziativ, kommutativ, hat ein neutrales Element bzgl. $V$ und für alle Elemente $v\in V$ Inverse in $V$
		\item $\otimes$ ist assoziativ, hat ein neutrales Element bzgl. $V$
		\item Elemente aus $K$ und $V$ sind distributiv
	\end{itemize}
	Dann nennen wir $V$ einen K-Vektorraum.
\end{theorem}

\begin{theorem}[Homomorphismus]
\label{Homomorphismus}
	Seien $V,W$ zwei K-Vektorräume und $f:V\rightarrow W$ eine Abbildung. Dann nennen wir $f$ einen Homomorphismus (oder linear) von $V$ nach $W$ wenn gilt:
	\begin{itemize}
		\item $\forall x,y\in V : f(x)+f(y) = f(x+y)$
		\item $\forall x\in V, \lambda\in K: f(\lambda\cdot K)=\lambda\cdot f(x)$
	\end{itemize}
\end{theorem}

