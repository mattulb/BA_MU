\chapter{Einleitung}
Ziel dieser Arbeit ist eine Erörterung der Vorteile von Anwendungen homomorpher Kryptosysteme. Homomorphe Kryptosysteme gliedern sich ihrerseits in semihomomorphe und vollhomomorphe Kryptosysteme. Während semihomomorphe Kryptosysteme eingeschränkt Operationen auf Chriffren ermöglichen - abhängig von dem gewählten Kryptosystem, ermöglichen vollhomomorphe Kryptosysteme die Berechnung jeder boolschen Funktion im Chiffreraum. Da letztere jedoch aufgrund hoher Laufzeiten weniger praktikabel sind greift man oft auf semihomomorphe Kryptosysteme zurück  [insert citation]. Dies führt in der Praxis zu verschiedenen, vom jeweiligen Anwendungsfall abhängigen Einsatz eines Kryptosystems.

Der Beitrag dieser Bachelorarbeit:\\ Es ist von vornerein nicht klar welche Gründe jemand gewählt hat ein Kryptosystem dem anderen vorzuziehen, insbesondere wenn sie den gleichen Operator im Chiffreraum zur Verfügung stellen. Weiter lässt sich mit einem Kryptosystem wie Paillier der Operator des Kryptosystems von Goldwasser-Micali simulieren [insert label]. Diese Arbeit will die Gründe für den Einsatz verschiedener homomorphen Kryptosysteme untersuchen und Informationen sammeln um die einzelnen Kryptosysteme zu kategorisieren als Referenz für den zukünfigen Einsatz homomorpher Kryptographie in Forschungsarbeiten.

