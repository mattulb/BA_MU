Diese Bachelorarbeit untersucht den vielfältigen Einsatz homomorpher Kryptosysteme in Forschungsarbeiten. Dabei wird erörtert, welches Kryptosystem zum Einsatz kommt und aus welchen Gründen ein Forscherteam sich für dieses entschieden hat. Es wird untersucht in welchem Anwendungskontext das Kryptosystem zum Einsatz kommt und ob die Malleabilität von homomorphen Kryptosystemen bei der Implementierung berücksichtigt wurde. Kurzgesagt ist Malleabilität ist die Fähigkeit auf Chiffren eine Funktion auszuführen, welche man im Klartext wiederfindet. Findet diese Veränderung unbemerkt statt, kann die Integrität von Rechenoperationen im Chiffreraum verletzt werden. Damit ist Malleabilität eine Eigenschaft die direkt aus der Homomorphieigenschaft resultiert. 

Die evaluierten Einsatzscenarien homorpher Kryptosysteme werden im Einzelnen kurz vorgestellt und herangezogen um Kriterien zu erstellen anhand welcher eine Kategorisierung der Kryptosysteme erfolgt. Dies geschieht im Hinblick darauf Dritten bei der Entscheidung für den Einsatz homomorpher Kryptographie für eigene Arbeiten zu unterstützen.
 