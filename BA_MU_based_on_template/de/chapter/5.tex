\chapter{Verwandte Arbeiten}
In \cite{maimut2012homomorphic} werden verschiedene partiell- und vollhomorphe Kryptosysteme vorgestellt als generelle Lösung für Vertraulichkeitsprobleme im Cloud Computing. Sie erwarten, dass homomorphe Kryptosysteme in der Zukunft schneller werden und somit ihre Relevanz für den Praxiseinsatz steigt. In ihrer Veröffentlichung werden die Ver- und Entschlüsselungsalgorithmen der verschiedener Kryptosysteme kurz vorgestellt und in Abschnitt 3 erste Beispiele für einen Einsatz in Cloud Computing, Sicheren Wahlsystemen und Privaten Informationsrückgewinnungssystemen erwähnt.

Der Einsatz von homomorpher Kryptographie geschieht nur unter dem Hintergrund der Sicherung von Vertraulichkeit. Weder eine Malleability der Kryptosysteme wird erörtert noch die Möglichkeit von Angreifern Schwächen in den vorgestellten Beispielen auszunutzen.

Der Artikel \cite{fontaine2007survey} dient als Übersicht und Einführung in Homomorphe Kryptographie für fachfremde Wissenschaftler. Die Autoren betonen ausdrücklich die Malleablität von homomorphen Kryptosystemen und stellen klar, dass diese geforderte Eigenschaft das Erreichen von IND-CCA2 Sicherheit unmöglich macht. Sie gehen auch darauf ein, dass die Forderung nach semantischer Sicherheit zur Folge hat, dass eingesetzte Kryptosysteme proballistisch sein müssen.
In einer anschließenden Vorstellung einzelner Kryptosysteme bleiben die Autoren jedoch der Vorstellung eines Anwendungskontexts schuldig.

Beide erwähnten Quellen erwähnen, die Unpraktiabilität von vollhomomorpher Kryptographie für den praktischen Einsatz wegen hoher Anforderung an vorhandene Rechenkapazität.